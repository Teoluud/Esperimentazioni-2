\documentclass[a4paper]{article}
\usepackage{graphicx} % Required for inserting images
\usepackage{physics}
\usepackage[margin=2cm]{geometry}
\usepackage[italian]{babel}
\usepackage{siunitx}
\usepackage{float}
\usepackage{hyperref}
\usepackage[shortlabels]{enumitem}
\sisetup{separate-uncertainty=true}
\hypersetup{hidelinks, linktoc=all}
\usepackage{subfig}
\usepackage{changepage}
\usepackage[toc,page]{appendix}
\usepackage{breqn}

\renewcommand\thesubfigure{\arabic{subfigure}}

\sisetup{per-mode=symbol}

\title{\textbf{Lenti}}
\author{Agostino Luca, Cafaro Alessandro, Gili Francesco, Gros Jacques Matteo\\ Turno AII - Gruppo 7\\A.A. 2024-2025}
\date{\today}

\begin{document}
    
\maketitle

\tableofcontents
\newpage

\section{Obiettivi della misura}
    Verificare la validità delle leggi sulle lenti sottili, misurandone le proprietà geometriche; in particolare:
    \begin{enumerate}
        \item Ricavare la distanza focale e l'ingrandimento di una lente biconvessa
        \item Ricavare la distanza focale di una lente piano-convessa
        \item Ricavare la distanza focale di una lente divergente
        \item Misurare la posizione dell'immagine di un sistema di due lenti convergenti non a contatto.
    \end{enumerate}
\section{Apparato sperimentale}
    \begin{itemize}
        \item Banco ottico (sensibilità: \SI{1}{\mm})
        \item Proiettore con illuminazione regolabile
        \item Diapositiva da proiettare
        \item Lenti di diverso tipo: biconvessa, piano-convessa, biconcava
        \item Schermo per visualizzare l'immagine
        \item Calibro (sensibilità: \SI{0.05}{\mm})
    \end{itemize}
    Abbiamo scelto di utilizzare come incertezza della scala graduata del banco ottico il valore di \SI{2}{\mm}, al posto della sua sensibilità, perché meglio rappresentativo della nostra precisione nell'effettuare la misurazione. D'ora in poi, dunque, quando si farà riferimento alla sensibilità dello strumento, si intenderà quella da noi ad esso associata.
\section{Presa dati}
    \subsection{Lente biconvessa}\label{sec:biconvessa}
        Abbiamo fissato la lente biconvessa su un supporto posto a distanza $p=\SI{140(2)}{\mm}$. Successivamente abbiamo compiuto 70 misure ripetute della posizione dell'immagine, spostando lo schermo finché questa non risultasse nitida, con l'accortezza di alternare destra e sinistra come direzioni di avvicinamento.
        %Dati.

        Ci aspettiamo che i due fuochi abbiano la stessa distanza dalla lente; l'abbiamo quindi ruotato di $\pi$ e ripetuto le misure utilizzando la stessa procedura e le medesime accortezze.

        Per entrambe le lenti abbiamo poi calcolato l'ingrandimento come il rapporto tra la distanza di due punti distinti sullo schermo e degli stessi sulla diapositiva.
        %Dati
        
    \subsection{Lente piano-convessa}    
    Abbiamo ripetuto le precedenti misurazioni su una lente convergente piano-convessa mantenendo inalterato il numero di misure e la procedura utilizzata.
    
    \subsection{Lente biconcava}
    Riutilizzando la lente biconvessa della Sezione \ref{sec:biconvessa}, abbiamo costruito un sistema ottico formato da quest'ultima e da una lente divergente biconcava, montandole in modo che fossero il più possibile vicine tra loro. Abbiamo posto le due lenti a distanza $p$ dall'oggetto e, come scritto sopra, abbiamo compiuto $N$ misurazioni di $q$.
    
    \subsection{Sistema di lenti}
    Utilizzando due lenti convergenti non a contatto, abbiamo misurato 10 volte il valore di $q_2$, ovvero la posizione dell'immagine del sistema.
\section{Analisi dati}
    \subsection{Lente biconvessa}
    Poiché abbiamo compiuto $N=70$ misure ripetute della stessa grandezza, affetta da errori casuali, ci aspettiamo che i dati si distribuiscano secondo un andamento gaussiano. Abbiamo dunque eseguito un fit dell'istogramma delle frequenze assolute dei dati.
    \begin{figure}[H]%
    	\centering
    	\subfloat[\centering Lente biconvessa]{\includegraphics[width=0.4\textwidth]{histo1.jpg}}%
    	\qquad
    	\subfloat[\centering Lente biconvessa ruotata]{\includegraphics[width=0.4\textwidth]{histo2.jpg}}%
    \end{figure}
    Di seguito riportiamo il test del $\chi^2$, con un livello di significatività del $5\%$:
    \[
    H_0: \text{la distribuzione gaussiana ben descrive quella dei dati sperimentali.}
    \]
    \begin{table}[H]
    	\centering
    	\begin{tabular}{|c|c|c|}
    		\hline
    		 & (1) & (2) \\ \hline
    		$\chi^2$ & 1.942 & 11.010 \\
    		d.o.f & 9 & 7 \\
    		$\chi^2_c$ & 16.919 & 14.067 \\ \hline
    	\end{tabular}
    	\label{tab:chi-quadro-biconvessa}
    \end{table}
    In entrambi i casi, $\chi^2\leq\chi^2_c$, dunque accettiamo l'ipotesi nulla.
    
    Siccome il fit gaussiano ben descrive la distribuzione dei dati, possiamo considerare $q=\mu$ con un livello di confidenza del 95\%:
    \begin{align*}
    	q_1 = \SI{385(2)}{\mm} && q_2 = \SI{393(2)}{\mm}
    \end{align*}
    
    dove come incertezza associata abbiamo tenuto la sensibilità dello strumento, in quanto la deviazione standard della media era inferiore.
    
    Successivamente abbiamo calcolato il valore del fuoco:
    \begin{align*}
    	f_1=\frac{p_1\cdot q_1}{p_1+q_1}=\SI{103(1)}{\mm} && f_2=\frac{p_2\cdot q_2}{p_2+q_2}=\SI{103(1)}{\mm}
    \end{align*}
    Come ci aspettavamo, le due distanze focali sono compatibili.
    
    L'ingrandimento vale invece:
    \begin{align*}
    	G_1=\frac{q_1}{p_1}=\SI{2.75(4)}{} && G_2=\frac{q_2}{p_2}=\SI{2.81(4)}{}
    \end{align*}
    Per verificare che il secondo ingrandimento sia compatibile con il rispettivo valore misurato, abbiamo effettuato un test di Gauss:
    \[
    H_0: \text{abbiamo estratto $G_2-G_{2m}$ da una distribuzione normale centrata intorno a $\mu=0$ con $\sigma=\sqrt{\sigma_{G_2}^2+\sigma_{G_{2m}}^2}=0.07$}
    \]
    \begin{table}[H]
    	\centering
    	\begin{tabular}{|c|c|}
    		\hline
    		z osservato ($z_o$) & 1.75 \\
    		livello di significatività ($\alpha$) & 5\% \\
    		z critico ($z_c$) & 1.96 \\ \hline
    	\end{tabular}
    	\label{tab:gauss-ingrandimento}
    \end{table}
    Poiché $z_o\leq z_c$, accettiamo l'ipotesi nulla $H_0$ con un livello di significatività del 5\%.
    \subsection{Lente piano-convessa}
    Abbiamo ripetuto il processo descritto nella sezione precedente per la lente piano-convessa.
    \begin{figure}[H]
    	\centering
    	\includegraphics[width=0.6\textwidth]{histo3.jpg}
    	\caption{Lente piano-convessa}
    	\label{fig:piano-convessa}
    \end{figure}
    Test del $\chi^2$ ($\alpha=5\%$):
     \[
    H_0: \text{la distribuzione gaussiana ben descrive i dati sperimentali.}
    \]
    \begin{table}[H]
    	\centering
    	\begin{tabular}{|c|c|}
    		\hline
    		$\chi^2$ & 23.424 \\
    		d.o.f & 11 \\
    		$\chi^2_c$ & 19.675 \\ \hline
    	\end{tabular}
    	\label{tab:chi-quadro-piano-convessa}
    \end{table}
    $\chi^2>\chi^2_c$, rifiutiamo dunque l'ipotesi nulla. Come si può osservare dall'istogramma delle frequenze assolute, questo è probabilmente dovuto ad un errore sistematico di sottostima della posizione dell'immagine.
    
    Scegliamo dunque come distanza lente-immagine il valor medio dei dati.
    \[
    q=\SI{204(2)}{\mm}
    \]
    La distanza focale è dunque pari a:
    \[
    f_3=\SI{79.4(8)}{\mm}
    \]
    \subsection{Lente biconcava}
    \begin{figure}[H]
    	\centering
    	\includegraphics[width=0.6\textwidth]{histo4.jpg}
    	\caption{Lente biconcava}
    	\label{fig:divergente}
    \end{figure}
    Test del $\chi^2$ ($\alpha=5\%$):
    \[
    H_0: \text{la gaussiana ben descrive la distribuzione dei dati sperimentali.}
    \]
    \begin{table}[H]
    	\centering
    	\begin{tabular}{|c|c|}
    		\hline
    		$\chi^2$ & 2.325 \\
    		d.o.f & 7 \\
    		$\chi^2_c$ & 14.067 \\ \hline
    	\end{tabular}
    	\label{tab:chi-quadro-divergente}
    \end{table}
    $\chi^2\leq\chi^2_c$, accettiamo dunque l'ipotesi nulla.
    \[
    q = \SI{274(4)}{\mm}
    \]
    Il fuoco della lente divergente è:
    \[
    f_D=\frac{f\cdot f_1}{f_1-f}=\SI{-7.5(7)e2}{\mm}
    \]
    \subsection{Sistema di lenti}
    Non abbiamo eseguito il fit di questa distribuzione di dati perché la grandezza del campione ($N=10$) era insufficiente. Il valore di $q$ è dunque dato dalla media aritmetica:
    \[
    q_m=\SI{318(2)}{\mm}
    \]
    Per verificare la compatibilità tra il valore misurato e quello calcolato a partire dalle proprietà geometriche note del sistema, abbiamo effettuato un test di Gauss:
    \[
    H_0: \text{abbiamo estratto $q_m-q_c$ da una distribuzione normale centrata intorno a $\mu=0$ con $\sigma=\sqrt{\sigma_{q_m}^2+\sigma_{q_c}^2}=\SI{2e1}{\mm}$}
    \]
    \begin{table}[H]
    	\centering
    	\begin{tabular}{|c|c|}
    		\hline
    		z osservato ($z_o$) & 0.326 \\
    		livello di significatività ($\alpha$) & 5\% \\
    		z critico ($z_c$) & 1.96 \\ \hline
    	\end{tabular}
    	\label{tab:gauss-sistema-lenti}
    \end{table}
    Poiché $z_o\leq z_c$, accettiamo l'ipotesi nulla $H_0$ con un livello di significatività del 5\%.
\section{Risultati e osservazioni conclusive}
Alla luce dei test statistici eseguiti, possiamo ritenerci soddisfatti del lavoro compiuto. Abbiamo infatti portato a termine con successo tutti gli obiettivi che ci eravamo proposti, ovvero:
\begin{enumerate}
	\item Tramite le approssimazioni di Gauss, misurare la distanza focale e l'ingrandimento delle lenti biconvessa e piano-convessa. Segnaliamo tuttavia la probabile presenza di un errore sistematico nell'acquisizione dei dati circa quest'ultima.
	\item Misurare la distanza focale di una lente divergente (biconcava), tramite l'utilizzo di un sistema ottico composto da due lenti a contatto (biconvessa-biconcava).
	\item Studiare un sistema ottico composto da due lenti non a contatto, utilizzando le proprietà ricavate nei punti precedenti.
\end{enumerate}
\begin{appendices}
    \section{Dati}
    \section{Calcoli}
\end{appendices}
\end{document}
